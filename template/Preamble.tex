\usepackage{xcolor, color}
\usepackage[dvipdfmx]{graphicx}
\usepackage{latexsym}
\usepackage{bmpsize}
\usepackage{url}
\usepackage{comment}

\usepackage{subfiles}
\usepackage{here}
\usepackage{breqn}

%subcaptionを使うためのやつ。
\usepackage[subrefformat=parens]{subcaption}


% \maketitle カスタマイズ
\usepackage{titling}
\pretitle{
	\vspace{-2.7cm} % タイトルを上に詰める
	\begin{center}
		\huge\sffamily % タイトル:hugeサイズ、ゴシック体
}
\posttitle{
	\end{center}
}
\preauthor{
	% \vspace{\baselineskip}
	\begin{center}
		\large\sffamily % 著者名:largeサイズ、ゴシック体
}
\postauthor{
	\end{center}
}
\predate{
	\begin{center}
		\large\sffamily % 日付:largeサイズ、ゴシック体
}
\postdate{
	\end{center}
}


% セクションのスタイル変更
\usepackage{titlesec}
\titleformat*{\section}{\Large\bfseries\sffamily}
\titleformat*{\subsection}{\normalsize\bfseries\sffamily}
% \titlespacing*{\section}{0pt}{*0}{0pt}
\titlespacing*{\subsection}{0pt}{8pt}{0pt}

% \makeatletter
% \def\section{\@startsection {section}{1}{\z@}{-1.5ex plus -1ex minus -.2ex}{1.5 ex plus .2ex}{\large\bf}}
% \def\subsection{\@startsection {subsection}{1}{\z@}{-1.5ex plus -3ex minus -.2ex}{1.3 ex plus .2ex}{\normalsize\bf}}
% \def\subsubsection{\@startsection {subsubsection}{1}{\z@}{-1.5ex plus -1ex minus -.2ex}{.3 ex plus .2ex}{\large \bf $\spadesuit$ }}
% \makeatother


% 参照マクロ
\newcommand{\fref}[1]{\textbf{図\ref{#1}}}
\newcommand{\Fref}[1]{\textbf{式(\ref{#1})}}
\newcommand{\tref}[1]{\textbf{表\ref{#1}}}


\usepackage{graphicx}
% listings 設定
% listings: ソースコードを表示するためのプラグイン


\usepackage{listings}
% コード部分の色スタイルの設定
\definecolor{bkg}{gray}{0.95}
\definecolor{def}{gray}{0.00}
\definecolor{com}{gray}{0.60}
\definecolor{key}{rgb}{0.00, 0.00, 0.75}
\definecolor{str}{rgb}{0.20, 0.50, 0.15}

% ソースコードを表示するときのキャプション名
\renewcommand{\lstlistingname}{コード}

% 書式設定
\lstset{
   % プログラミング言語
   language={python},
   % 背景色
   backgroundcolor={\color{bkg}},
   % 基本の文字スタイル
   basicstyle={\small\ttfamily\color{def}},
   % 変数の文字スタイル
   identifierstyle={\small\ttfamily\color{def}},
   % コメントの文字スタイル
   commentstyle={\color{com}},
   % 予約語の文字スタイル
   keywordstyle={\bfseries\color{key}},
   % 非予約語の文字スタイル (よくわからない)
   ndkeywordstyle={\small\color{def}},
   % 文字列リテラルのスタイル
   stringstyle={\bfseries\color{str}},
   % 枠線の設定
   % t, r, b, l: それぞれ上、右、下、左の1本線
   % T, R, B, L: それぞれ上、右、下、左の2本線
   frame={tlRB},
   % 長い文を改行するかどうか
   breaklines=true,
   % 横幅間隔の調整
   columns=[l]{fullflexible},
   % 左右のマージン
	 xrightmargin=0\zw,
   xleftmargin=1\zw,
   framexleftmargin=3pt,
   % 行番号の位置
   numbers=left,
   % 行番号のスタイル
   numberstyle={\ttfamily\small},
   % 行番号とコード本文の間の空白
	 numbersep=1\zw,
   % 行番号の刻み
   stepnumber=1,
   % コメント行の継続の設定
	morecomment=[l]{//}
}
\newcommand{\cref}[1]{\textbf{\lstlistingname\ref{#1}}}
